\documentclass[12pt]{article}
\usepackage{lingmacros}
\usepackage{tree-dvips}
\usepackage{amsmath}
\usepackage{accents}
\newcommand{\ubar}[1]{\underaccent{\bar}{#1}}
\usepackage{hyperref}
\hypersetup{
    colorlinks=true,
    linkcolor=blue,
    filecolor=magenta,      
    urlcolor=cyan,
}
\begin{document}

{\centering
 \textbf{SIE 550 (Linear) Systems Theory\\Homework \#6 - Due date: Wednesday, May 3, 2017 \newline}\par
 
} 

\noindent
\textbf{Problem 1:} Apply the corollary of Theorem 9.1 for system ({\em Linear Systems Theory}) Chapter 9, Problem 6 (page 458)
	
$$
\dot{\bar{x}}=
\begin{pmatrix}
	1 & 1 \\
	2 & 2
\end{pmatrix}
\bar{x}+
\begin{pmatrix}
	1 \\
	0
\end{pmatrix} u
$$
and polynomial $p(\lambda)=\lambda^2+2\lambda+1$. Select $c^T=(1, 0)$.\\
Confirm your hand calculation via MATLAB (use "place" or "acker" command depending on the situation). Assume this is a regulation problem and provide a MATLAB simulation that shows the behavior of the states and of the output of the systems form the following initial conditions $x(0)=[1,1]^T$\\

\noindent
\textbf{Problem 2:} Apply the corollary of Theorem 9.1 for system ({\em Linear Systems Theory}) Chapter 9, Problem 7 (page 458)
$$
\dot{\bar{x}}=
\begin{pmatrix}
	0 & 1 & 2 \\
	0 & 1 & 1 \\
	0 & 0 & 2
\end{pmatrix}
\bar{x}+
\begin{pmatrix}
	1 \\
	2 \\
	1
\end{pmatrix} u
$$
and polynomial $p(\lambda)=\lambda^3+1$. Select $c^T=(1, 1, 1)$.\\
Confirm your hand calculation via MATLAB (use "place" or "acker" command depending on the situation). Assume this is a regulation problem and provide a MATLAB simulation that shows the behavior of the states and of the output of the systems form the following initial conditions $x(0)=[1,1,1]^T$\\

\noindent
\textbf{Problem 3:} Apply the corollary of Theorem 9.1 for system ({\em Linear Systems Theory}) Chapter 9, Problem 9 (page 458)
$$
\dot{\bar{x}}=
\begin{pmatrix}
	2 & 1 \\
	0 & 1 \\
\end{pmatrix}
\bar{x}+
\begin{pmatrix}
	1 \\
	1
\end{pmatrix} u
$$
and polynomial $p(\lambda)=\lambda^2+\lambda+1$. Select $c^T=(1, 1)$.\\
Confirm your hand calculation via MATLAB (use "place" or "acker" command depending on the situation). Assume this is a regulation problem and provide a MATLAB simulation that shows the behavior of the states and of the output of the systems form the following initial conditions $x(0)=[1,1,0]^T$\\

\noindent
\textbf{Problem 4:} Construct an observer (9.12) for system ({\em Linear Systems Theory}) Chapter 9, Problem 14 (page 459)
$$
\dot{\bar{x}}=
\begin{pmatrix}
	0 & 0 & 0 \\
	0 & 1 & 0 \\
	0 & 0 & 2
\end{pmatrix}
\bar{x}+
\begin{pmatrix}
	1 \\
	1 \\
	1
\end{pmatrix} u
$$
$$y=(1,1,1)\bar{x}$$
Confirm your hand calculation via MATLAB (use "place" or "acker" command depending on the situation). Show the behavior of the observer error dynamics by providing a MATLAB simulation of $\dot{\bar{e}}=(A-K_eC)\bar{e}$ assuming the following initial conditions $\bar{e}(0)=[0,0.5,1]^T$ \\

\noindent
\textbf{Problem 5:} Construct an observer (9.12) for system ({\em Linear Systems Theory}) Chapter 9, Problem 15 (page 459)
$$
\dot{\bar{x}}=
\begin{pmatrix}
	2 & 1 \\
	0 & 1 \\
\end{pmatrix}
\bar{x}+
\begin{pmatrix}
	1 \\
	1
\end{pmatrix} u
$$
$$y=(1,1)\bar{x}$$
Confirm your hand calculation via MATLAB (use "place" or "acker" command depending on the situation). Show the behavior of the observer error dynamics by providing a MATLAB simulation of $\dot{\bar{e}}=(A-K_eC)\bar{e}$ assuming the following initial conditions $\bar{e}(0)=[0.5, 0.5]^T$
\end{document}
