\documentclass[12pt]{article}
\usepackage{lingmacros}
\usepackage{tree-dvips}
\usepackage{amsmath}
\usepackage{accents}
\newcommand{\ubar}[1]{\underaccent{\bar}{#1}}
\usepackage{hyperref}
\hypersetup{
    colorlinks=true,
    linkcolor=blue,
    filecolor=magenta,      
    urlcolor=cyan,
}
\begin{document}

\title{SIE 550 HW 3 Meeting Notes}
\date{2/27/17}

\maketitle

Fork and contribute to these notes on: \url{https://github.com/eggsactly/SIE550}

\section*{Problem 1} 
We feel comfortable with problem 1, problem 3 looks like it will be hard because it is time variant. 

\section*{Problem 2} 
\subsection*{2.1}
Finding the eigenvalues, solve this equation:
$$det(A-I\lambda)=0$$
Look at the real part of all the eigenvalues, if all the values are less than zero the system is asymptotically stable.\\
If any values are greater than zero then the system is unstable. 

\subsection*{2.2}
What is it asking us to do?

\subsubsection*{2.2a}
Plug in 
$$P=\begin{bmatrix} p_{11} & p_{12} \\
p_{12} & p_{22}
\end{bmatrix}$$ 

and 

$$\ubar{x}=\begin{bmatrix} x_{1} \\
x_{2}
\end{bmatrix}$$ 

to

$$V(\ubar{x})=\ubar{x}^TP\ubar{x}$$

You will get a scalar out of that equation if you do the matrix multiplication. Then look at the two equations. 
$p_{11}>0$ 
and
$p_{11}p_{22}-p_{12}^2>0$ \\

For the second equation to be true, $p_{22}>0$ 

Then you would need to explain that $p_{11}x_{1}^2$ and $p_{22}x_{2}^2$ dominate $p_{12}x_1x_2$ in all values for $\ubar{x}$. 

\subsubsection*{2.2b}
Find P in $A^TP+PA=-Q$ where 
$A=\begin{bmatrix} 0 & -1 \\
1 & -1
\end{bmatrix}$
$Q=\begin{bmatrix} 1 & 0 \\
0 & 1
\end{bmatrix}$

You can still assume that $P=\begin{bmatrix} p_{11} & p_{12} \\
p_{12} & p_{22}
\end{bmatrix}$

At the end, you will have solved for $p_{11}$, $p_{12}$, $p_{22}$. Once you have solved for P, find the eigenvalues for P. If all the eigenvalues for P are positive then the matrix is positive definite. 

\section*{Problem 3}
\subsection*{3.1}
Look at page 20 on the week 3 lecture notes. 

We know that 
$$A(t)=\begin{bmatrix} -1 & e^{2t} \\
0 & -1
\end{bmatrix}=\begin{cases}\dot{x_1}=-x_1+e^{2t}x_2\\ 
\dot{x_2}=-x_2\\ 
x(t_0)=[x_{10}, x_{20}]^T\\
\end{cases}$$ 

$$A(t)=\begin{bmatrix} \phi_{11}(t,t_0) & \phi_{12}(t,t_0) \\
0 & \phi_{22}(t,t_0)
\end{bmatrix}$$

$\phi_{ii}(t,t_0)$ can be solved, but there is no equation for anything off the diagonal, regardless we can still solve.

Solve for each line:
$$x_2(t)=e^{-(t-t_0)}x_{10}$$
$$x_1(t)=\phi_{11}(t,t_0)x_{10}+\int_{t_0}^{t}\phi_{11}(t,\tau)a_{21}(\tau)x_2(\tau)x_{20}d\tau$$

This will give you the solution for each cell:
$$\begin{bmatrix} x_1 \\
x_2
\end{bmatrix} = \begin{bmatrix} \phi_{11}(t,t_0) & \phi_{12}(t,t_0) \\
0 & \phi_{22}(t,t_0)
\end{bmatrix} \begin{bmatrix} x_{10} \\
x_{20}
\end{bmatrix} $$

Which is a moot point because 
$$\begin{bmatrix} x_{10} \\
x_{20}
\end{bmatrix} = \begin{bmatrix} 0 \\
0
\end{bmatrix} $$
Because the initial conditions for this problem are all zero.
\end{document}
