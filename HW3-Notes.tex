\documentclass[12pt]{article}
\usepackage{lingmacros}
\usepackage{tree-dvips}
\usepackage{amsmath}
\usepackage{accents}
\newcommand{\ubar}[1]{\underaccent{\bar}{#1}}
\usepackage{hyperref}
\hypersetup{
    colorlinks=true,
    linkcolor=blue,
    filecolor=magenta,      
    urlcolor=cyan,
}
\begin{document}

\title{SIE 550 HW 3 Meeting Notes}
\date{3/4/17}

\maketitle

Fork and contribute to these notes on: \url{https://github.com/eggsactly/SIE550}

\section*{Problem 1} 
We feel comfortable with problem 1, problem 3 looks like it will be hard because it is time variant. 

\section*{Problem 2} 
\subsection*{2.1}
Finding the eigenvalues, solve this equation:
$$det(A-I\lambda)=0$$
Look at the real part of all the eigenvalues, if all the values are less than zero the system is asymptotically stable.\\
If any values are greater than zero then the system is unstable. 

\subsection*{2.2}
What is it asking us to do?

\subsubsection*{2.2a}
Plug in 
$$P=\begin{bmatrix} p_{11} & p_{12} \\
p_{12} & p_{22}
\end{bmatrix}$$ 

and 

$$\ubar{x}=\begin{bmatrix} x_{1} \\
x_{2}
\end{bmatrix}$$ 

to

$$V(\ubar{x})=\ubar{x}^TP\ubar{x}$$

You will get a scalar out of that equation if you do the matrix multiplication. Then look at the two equations. 
$p_{11}>0$ 
and
$p_{11}p_{22}-p_{12}^2>0$ \\

For the second equation to be true, $p_{22}>0$ 

Then you would need to explain that $p_{11}x_{1}^2$ and $p_{22}x_{2}^2$ dominate $p_{12}x_1x_2$ in all values for $\ubar{x}$. 

\subsubsection*{2.2b}
Find P in $A^TP+PA=-Q$ where 
$A=\begin{bmatrix} 0 & -1 \\
1 & -1
\end{bmatrix}$
$Q=\begin{bmatrix} 1 & 0 \\
0 & 1
\end{bmatrix}$

You can still assume that $P=\begin{bmatrix} p_{11} & p_{12} \\
p_{12} & p_{22}
\end{bmatrix}$

At the end, you will have solved for $p_{11}$, $p_{12}$, $p_{22}$. Once you have solved for P, find the eigenvalues for P. If all the eigenvalues for P are positive then the matrix is positive definite. 

\section*{Problem 3}
\subsection*{3.1}
Look at page 20 on the week 3 lecture notes. 

We know that
$$A(t)=\begin{bmatrix} -1 & e^{2t} \\
0 & -1
\end{bmatrix}=\begin{cases}\dot{x_1}=-x_1+e^{2t}x_2\\ 
\dot{x_2}=-x_2\\ 
x(t_0)=[x_1(0), x_2(0)]^T\\
\end{cases}$$ 

Solve for each line:
$$\dot{x_2}=-x_2\implies x_2=e^{-t}x_2(0)$$
$$\dot{x_1}=-x_1+e^{2t}e_2=-x_1+e^{2t}e^{-t}x_2(0)$$
$$\dot{x_1}+x_1=e^tx_2(0)$$
$$\dot{x_1}=x_{1_H}+x_{1_P}$$
$$\dot{x_1}+x_1=0\implies x_{1_H}=-e^tx(0)$$ (Separation of variables)
$$x_{1_P}=Ae^t\implies Ae^t+Ae^t=-e^tx_2(0)\implies A=\frac{x_2(0)}{2}$$
$$x_1=-e^tx_1(0)+\frac{x_2(0)}{2}e^t$$
$$\bar{x}=\begin{bmatrix} -e^t & \frac{e^t}{2} \\
0 & e^{-t}
\end{bmatrix}
\begin{bmatrix}x_1(0)\\ 
x_2(0)\\ 
\end{bmatrix}$$ 

\subsection*{3.2}
We do not know what this question is asking as of 3/2/17.

\section*{Problem 7}
In lecture on 2/21/17 Furfaro said that the following problem shows up on the homework:
$$\begin{cases}\dot{x_1}=-x_1-g(t)x_2\\ 
\dot{x_2}=x_1-x_2\\
\end{cases}$$ 

$$0\leq g(t)\leq k$$
$$\dot{g(t)} \leq g(t), \forall t \leq 0$$

Choose a good Lyapunov function. 
$$v(\bar{x},t)=x_1^2+(1+g(t))x_2^2$$

1. is a good candidate\\
$$\alpha(||\bar{x}||)\leq v(\bar{x},t)\leq \beta(||\bar{x}||)$$

$$(x_1^2+x_2^2)\leq x_1^2+(1+g(t))x_2^2\leq x_1^2+(1+k)x_2^2$$

$$\leq(1+k)(x_1^2+x_2^2)$$

$$(x_1^2+x_2^2)\leq v(\bar{x},t)\leq (1+k)(x_1^2+x_2^2)$$
$$(x_1^2+x_2^2)\leq v(\bar{x},t)\leq (1+k)||\bar{x}||^2$$
$$\alpha(||\bar{x}||)=(x_1^2+x_2^2)$$
$$\beta=(||\bar{x}||)$$
$$c_1||\bar{x}||^2\leq v(\bar{x},t)\leq c_2||\bar{x}||^2$$
$c_1=1$, $c_2=1+k$
$$\dot{v}=2x_1\dot{x_1}+\dot{g(t)}x_2^2+(1+g(t))x_2\dot{x_2}$$
$$\dot{v}(\bar{x},t)=-2x_1^2+2x_1x_2-[2+2g(t)-\dot{g}(t)]x_2^2$$
$$\dot{v}=-[2x_1^2-2x_1x_2+[2+2g(t)-\dot{g}(t)]x_2^2]$$
$$\dot{v}=\bar{x}^TP\bar{x}=\begin{bmatrix} x_1 & x_2 \end{bmatrix}
\begin{bmatrix} p_{11} & p_{12} \\
p_{21} & p_{22} \end{bmatrix}\begin{bmatrix} x_1 \\ x_2 \end{bmatrix} = (p_{11}x_1+p_{12}x_2)x_1+(p_{12}x_1+p_{22}x_2)x_2$$
Because $\dot{g}\leq g$
$$2+2g(t)-\dot{g}(t)\geq 2+2g(t)=g(t)$$
$$2+2g(t)-\dot{g}(t)\geq 2+g(t)$$
$$0\leq g(t) \leq k$$
Conclusion:
$$\dot{v}\leq -2x_1^2+2x_2x_2+[\alpha+2g(t)-\dot{g}(t)]$$
$$\dot{v}\leq -2x_1^2+2x_1x_2-2x_2^2$$
$$\dot{v}\leq -2||\bar{x}||^2$$

\end{document}


