\documentclass[12pt]{article}
\usepackage{lingmacros}
\usepackage{tree-dvips}
\usepackage{amsmath}
\usepackage{accents}
\newcommand{\ubar}[1]{\underaccent{\bar}{#1}}
\usepackage{hyperref}
\hypersetup{
    colorlinks=true,
    linkcolor=blue,
    filecolor=magenta,      
    urlcolor=cyan,
}
\begin{document}

\section*{Definitions}

\begin{itemize}

\item Basis: The basis of a set of vectors is a subset where all vectors of the set can be represented as a linear combination of vectors in the subset. 

\item Controllable: A dynamic system with initial condition $x(t_0)=x_0$ is said to be controllable to state $x_1$ at $t_1(>t_0)$ if there exists an input $u(t)$ such that $x(t_1)=x_1$ ({\em Linear Systems Theory}, Page 243)  .

\item Decrescent: $v(\bar{x},t)$ is decrescent if $v(\bar{0},t)=0$ and if $\exists$ \\ a time-invariant positive definite function $v_e(\bar{x})$ such that $\forall t\geq0$, $v(\bar{x},t)\leq v_e(\bar{x})$.

\item Dimension: The number of vectors in the basis.

\item Hurowitz: When all eigenvalues of A are such that $\mathcal{R}_e(\lambda_i)<0$ A is called Hurowitz. 

\item Inner Sum (Direct Sum): Represented with the symbol $\oplus$ is a coordinate-wise sum. For example consider the sets $a=(x_1,y_1)$ and $b=(x_2, y_2)$, $a\oplus b=(x_1+x_2,y_1+y_2)$.

\item Invariant: A set $G$ is an {\em invariant set} for a dynamic system if every system trajectory which starts in $G$ stays in $G$ for all future times. (Stability concepts (II) pages 1)

\item Limit Cycle: A limit cycle in the phase-plane of a second-order system is an isolated closed trajectory. 

\item Non-Singular: A matrix is non-singular if an inverse exists for the matrix. A square matrix is non-singular {\em if-and-only-if} its determinant is non-zero.

\item Null Space: The {\em Null Space} of a matrix ${\cal N}(A)$ is the set of vectors $\bar{x}$ where $A\bar{x}=\bar{0}$

\item Positive Definite: For function $V(\bar{x})>0, \forall x | x\neq0$ (Stability Concepts (I), Page 9). A matrix is positive definite if all the eigen-values are greater than zero. 

\item Positive Semi-Definite: For function $V(\bar{x})\geq0, \forall x | x\neq0$ A matrix is positive semi-definite if all the eigen-values are greater than or equal to zero. 

\item Range: The {\em Range} of a matrix ${\cal R}(A)$ is the span of all the column vectors in matrix $A$.

\item Rank: The maximum number of linearly independent column (or row) vectors in a matrix. ({\em Linear Systems Theory}, Page 257) 

\item Semi-Definite Negative: For function $V(\bar{x})\leq0$ (Stability Concepts (II), Page 3). 

\item Span: The set of all linear combinations of vectors in a set.

\item Transition Matrix (Fundamental Matrix): $\phi(t,\tau)=e^{A(t-\tau)}$

\end{itemize}


\newpage
\section*{Stability Theorems}
\subsection*{Stability of Autonomous Systems, $\dot{\bar{x}}=\bar{f}(\bar{x})$}
\begin{itemize}
\item Stable (LaSalle theorem) (Stability Concepts (I), page 11): a time invariant system is {\em stable} when it meets the following criteria. There exists a {\em Lyapunov function} where:
	\begin{itemize}
	\item $V(\bar{x})>0, \forall x \in D/\{\bar{0}\}$
	\item $V(\bar{0})=\bar{0}$
	\item $V(\bar{x})\leq0, \forall x \in D$
	\end{itemize}
\item Asymptotically Stable if:
	\begin{itemize}
	\item The criteria for stable are met and
	\item $V(\bar{x})<0, \forall x \in D/\{\bar{0}\}$
	\end{itemize}
\item Globally Asymptotically Stable if:
	\begin{itemize}
	\item The criteria for Asymptotically Stable are met and
	\item $V(\bar{x})\to\infty$ as $||\bar{x}||\to\infty$
	\end{itemize}
\end{itemize}

\newpage
\subsection*{Stability of Non-Autonomous Systems, $\dot{\bar{x}}=\bar{f}(\bar{x},t)$}
\begin{itemize}
\item Stable (Barbalat's Lemma) (Stability Concepts (II), page 10): a time variant system is {\em stable} when it meets the following criteria. There exists a {\em Lyapunov function} where:
	\begin{itemize}
	\item $V(\bar{x},t)>0$ is positive definite
	\item $\dot{V}(\bar{x},t)$ is negative semi-definite 
	\end{itemize}
\item Uniformly Stable where:
	\begin{itemize}
	\item The Criteria for Stable are met and
	\item $V(\bar{x},t)$ is {\em Decrescent}
	\end{itemize}
\item Uniformly Asymptotically Stable
	\begin{itemize}
	\item The Criteria for Uniform Stability are met and
	\item $\dot{V}(\bar{x},t)<0,\forall x \in D/\{\bar{0}\}$
	\end{itemize}
\item Globally Uniformly Asymptotically Stable:
	\begin{itemize}
	\item The Criteria for Uniform Asymptotically Stable are met and
	\item $V(\bar{x})\to\infty$ as $||\bar{x}||\to\infty$
	\end{itemize}
\item Exponential Stability (Stability Concepts (II), page 20):
	\begin{itemize}
	\item $V(\bar{x},t)\geq\alpha(||\bar{x}||)>0$
	\item $\dot{V}(\bar{x},t)\leq-\gamma(||\bar{x}||)$
	\item $V(\bar{x},t)\leq\beta(||\bar{x}||)$
	\item such that:
		\begin{itemize}
		\item $\alpha(||\bar{x}||)=k_1||\bar{x}||^a$
		\item $\beta(||\bar{x}||)=k_2||\bar{x}||^a$
		\item $\gamma(||\bar{x}||)=k_3||\bar{x}||^a$
		\end{itemize}
	\end{itemize}
\item Globally Exponentially Stable
	\begin{itemize}
	\item Exponential Stability criteria apply and:
	\item $c_1||\bar{x}||^2\leq V(\bar{x},t)\leq c_2||\bar{x}||^2$
	\end{itemize}
\item Finite Time Stability (Week 7 Lecture notes, page 7)
	\begin{itemize}
	\item There exist a continuously differentiable Lyapunov Function and a function $R(z)$ where $\dot{V}(\bar{x},t)\leq-R(V(\bar{x},t))$
	\item $\int_0^E\frac{dz}{R(z)}<+\infty$ for $E>0$
	\end{itemize}
\item Uniform Finite Time Stability
	\begin{itemize}
	\item The criteria for Finite Time Stability are met and: 
	\item $V(\bar{x},t)$ is {\em decrescent}.
	\end{itemize}
\item Globally Finite Time Stable 
	\begin{itemize}
	\item The criteria for Uniform Finite Time Stability are met and: 
	\item $V(\bar{x},t)$ is {\em Radially Unbounded}
	\end{itemize}
\item Finite Time Stability Converse (Week 7 Lecture notes, page 16)
	 \begin{itemize}
	 \item A system is {\em not} Finite time stable if:
	 \item There exist a continuously differentiable Lyapunov Function and a function $S(z)$ where $\dot{V}(\bar{x},t)\leq-S(V(\bar{x},t))$
	 \item $\int_0^E\frac{dz}{S(z)}=+\infty$ for $E>0$
	 \end{itemize}

\end{itemize}

\section*{Jordan Form}
\subsection*{Finding Eigenvectors For Defective Matricies}
Given that $\lambda$ is repeated $n$ times. $v_i$ can be found with:
$$det(A-\lambda I)\bar{v_1}=\bar{0}$$
$$det(A-\lambda I)\bar{v_2}=\bar{v_1}$$
$$det(A-\lambda I)\bar{v_3}=\bar{v_2}$$
And so on...
\subsection*{Finding Solution With Jordan Blocks}
If you have a Jordan Block, a solution will be in the form:
$$e^{J_it}=e^{(\lambda_iI+N_i)t}=e^{\lambda_it}e^{N_it}$$
Where $e^{\lambda_it}$ is a scalar value and $e^{N_it}$ is a matrix. $N_i$ is in the form:

$$
N_i=\begin{bmatrix} 
0 & 1 & 0 & \dots & 0 \\
0 & 0 & 1 & \dots & 0 \\
\vdots & \vdots & \vdots & \ddots & \vdots \\
0 & 0 & 0 & 0 & 1 \\
0 & 0 & 0 & 0 & 0\\
\end{bmatrix}
$$

$e^{N_it}$ can be found with the following equation.

$$e^{N_it}=\begin{bmatrix} 1 & t & \frac{t^2}{2} & \dots & \frac{t^{n-1}}{(n-1)!} \\
0 & 1 & t & \dots & \frac{t^{n-2}}{(n-2)!} \\
\vdots & \vdots & \vdots & \ddots & \vdots \\
0 & 0 & 0 & 1 & t \\
0 & 0 & 0 & 0 & 1\\
\end{bmatrix}$$ 

\newpage
\section*{Solutions to Time Varying Linear Systems (Non-Autonomous)}
Consider systems in the form 
$$ 
\begin{cases}
\dot{\bar{x}}=A(t)\bar{x} \\
\bar{x}(t_0)=\bar{x}_0
\end{cases}
$$
where $\{a_{ij}(t)\}$ are continuous on $[t_0,t_0+T]$ \\

\noindent
The solution is in the form (systems theory week 3 page 17):
$$\bar{x}(t)=\Phi(t,t_0)\bar{x}_0$$

\noindent
where the {\em Transition Matrix} is in the form:

$$ \Phi(t,t_0) = I + \int_{t_0}^t A(\tau_1)d\tau_1+\int_{t_0}^tA(\tau_1)\int_{t_0}^{\tau_1}A(\tau_2)d\tau_2d\tau_1+\dots$$

\noindent
Note that for {\em Linear Time Invariant} Systems $\dot{\bar{x}}=A\bar{x}$ 
$$\bar{x}(t)=e^{At}\bar{x}(0)\rightarrow \Phi(t,0)=e^{At}$$

\subsection*{Properties of $ \Phi(t,\tau)$}

\begin{itemize}
	\item $\Phi(t,t) = I$
	\item $\frac{\partial}{\partial t}\Phi(t,\tau)=A(t)\Phi(t,\tau)$
	\item $\frac{\partial}{\partial t}\Phi(t,\tau)=-\Phi(t,\tau)A(t)$
	\item $\Phi(t,\tau)=\Phi(t,\sigma)\Phi(\sigma,\tau)$
	\item $\Phi(t,\tau)$ is non-singular for all t, $\tau$
	\item $\Phi^{-1}(t,\tau)=\Phi(\tau,t)$
\end{itemize}

\subsection*{Special Cases}
If $A(t)$ is a triangular matrix:

$$ A(t) = 
\begin{bmatrix}
a_{11}(t) & a_{12}(t) & a_{13}(t) \\
0 & a_{22}(t) & a_{23}(t) \\
0 & 0& a_{33}(t)\\
\end{bmatrix}
$$
Then so too is $\Phi(t,t_0)$:
$$\Phi(t,t_0) = 
\begin{bmatrix}
\phi_{11}(t,t_0) & \phi_{12}(t,t_0) & \phi_{13}(t,t_0)\\
0 & \phi_{22}(t,t_0) & \phi_{23}(t,t_0) \\
0 & 0 & \phi_{33}(t,t_0)\\
\end{bmatrix}
$$

\noindent
The diagonal transition functions can be found with this equation:
$$\phi_{ii}(t,t_0)=e^{\int_{t_0}^ta_{ii}(\tau)d\tau}$$

\section*{Controllability}
\subsection*{Theorem 5.1 ({\em Linear Systems Theory} Page. 249)}
The continuous linear system is controllable from any initial state $x(t_0)=x_0$ to an arbitrary state $x_1$ at time $t_1>t_0$ if and only if matrix $W(t_0,t_1)$ is nonsingular.

\subsection*{Controllability Gramian}
$$W(t_0,t_1)=\int_{t_0}^{t_1}\phi(t_0,\tau)B(\tau)B^T(\tau)\phi^T(t_0,\tau)d\tau$$

\subsection*{Theorem 5.2 ({\em Linear Systems Theory} Page. 251)}
Matrix $W(t_0,t_1)$ satisfies the following properties:
\renewcommand{\labelenumiii}{\Roman{enumii}}
\begin{enumerate}
  \item {\em It is symmetric.}
  \item {\em It is positive semi-definite}. 
  \item $(\partial/\partial t)W(t,t_1)=A(t)W(t,t_1)+W(t,t_1)A^T(t)-B(t)B^T(t)$, \\$W(t_1,t_1)=0$
  \item $W(t_0,t_1)=W(t_0,t)+\phi(t_0,t)W(t,t_1)\phi^T(t_0,t)$
\end{enumerate}

\subsection*{Theorem 5.3 ({\em Linear Systems Theory} Page. 254)}
Assume that with some positive integer $q$, $B(t)$ is q-times continuously differentiable, and $A(t)$ is $(q-1)$-times continuously differentiable on the interface $[t_0,t_1]$, furthermore for some $t*\in [t_0,t_1]$,
$$rank(K_0(t*),K_1(t*),\dots,K_q(t*))=n$$
Then the following system
$$\dot{\bar{x}}=A(t)\bar{x}+B(t)u,\quad \bar{x}(t_0)=x_0$$
is controllable. 

\subsection*{Theorem 5.4 ({\em Linear Systems Theory} Page. 258)}
The time-invariant continuous linear system is completely controllable if and only if the rank of the controllability matrix $K$ equals n.
$$K=(B,AB,A^2B,\dots,A^{n-1}B)$$
$rank(K)=n$ $\rightarrow$ Fully controllable (completly).
 

\newpage
\section*{Observability}

\subsection*{Equation 5.6 ({\em Linear Systems Theory} Page. 296)}
$$M(t_0,t_1)=\int_{t_0}^{t_1}\phi^T(\tau,t_0)C^T(\tau)C(\tau)\phi(\tau,t_0)d\tau$$

\subsection*{Theorem 6.1 ({\em Linear Systems Theory} Page. 296)}
It is possible to determine $x_0$ with in an additive constant vector, which is in $N(M(t_0,t_1))$. If $M(t_0,t_1)$ is nonsingular, then $x_0$ can be determined uniquely.

\subsection*{Theorem 6.2 ({\em Linear Systems Theory} Page. 299)}
Matrix $M(t_0,t_1)$ satisfies the following properties:
\renewcommand{\labelenumiii}{\Roman{enumii}}
\begin{enumerate}
  \item {\em It is symmetric.}
  \item {\em It is positive semi-definite}. 
  \item $(\partial/\partial t)M(t,t_1)=-A(t)M(t,t_1)-M(t,t_1)A(t)-C^T(t)C(t)$, \\$M(t_1,t_1):0$
  \item $M(t_0,t_1)=M(t_0,t)+\phi(t,t_0)M(t,t_1)\phi^T(t,t_0)$
\end{enumerate}

\subsection*{Theorem 6.3 ({\em Linear Systems Theory} Page. 300)}
The time-invariant continuous linear system is observable for arbitrary $t_1>t_0$ if and only if the rank of the observability matrix $L$ equals n.

$$L=\begin{bmatrix}
C \\
CA \\
CA^2 \\
CA^{m-1}
\end{bmatrix}
$$
$rank(L)=n$ $\rightarrow$ Fully observable.
\subsection*{Theorem 6.4 ({\em Linear Systems Theory} Page. 302)}
Assume that the rank r of matrix $L$ is less than n. Then there exists a nonsingular matrix $T$ such that

$$\bar{A}=TAT^{-1}=\begin{pmatrix}
\bar{A}_{11} & O \\
\bar{A}_{21} & \bar{A}_{22}
\end{pmatrix}
$$

$$\bar{B}=TB=\begin{pmatrix}
\bar{B}_{1}\\
\bar{B}_{2}
\end{pmatrix}
$$

$$\bar{A}=CT^{-1}=\begin{pmatrix}
\bar{C}_{1} & O 
\end{pmatrix}
$$

\noindent
where the sizes of matrices $\bar{A}_{11}$, $\bar{A}_{21}$, $\bar{A}_{22}$ are $r\times r, (n-r)\times r, (n-r)\times (n-r),$ respectively, and $\bar{B}_1$ has r rows and $\bar{C}_1$ has r columns. Furthermore,
\renewcommand{\labelenumiii}{\Roman{enumii}}
\begin{enumerate}
  \item system $(\bar{A}_{11}, \bar{B}_1, \bar{C}_1)$ is completely observable, and
  \item the transfer function of systems $(A, B, C)$ and $(\bar{A}_{11}, \bar{B}_1, \bar{C}_1)$ coincide.
\end{enumerate}

\subsection*{Theorem 6.5 ({\em Linear Systems Theory} Page. 303)}
System $(A, B, C)$ is completely observable if and only if matrix $A$ has not eigenvector $q$ that is orthogonal to the rows of $C$.

\subsection*{Theorem 6.6 ({\em Linear Systems Theory} Page. 304)}
It is possible to determine $x_0$ within an additive constant vector, which is in $N(M(0,t_1))$. If $M(0,t_1)$ is nonsingular, then $x_0$ can be determined uniquely. 

\subsection*{Theorem 6.7 ({\em Linear Systems Theory} Page. 304)}
The time-invariant discrete linear system is observable at arbitrary $t_1\geq n$ if and only if the rank of the observability matrix $L$ equals n.

\subsection*{Primal vs. Dual ({\em SIE550\_FinalReview} Page 2)}
$$
\begin{cases}
	\dot{\bar{x}}=A\bar{x}+B\bar{u} \\
	\bar{y}=C\bar{x}
\end{cases}
\longleftrightarrow
\begin{cases}
	\dot{\bar{z}}=A^T\bar{z}+C^T\bar{v} \\
	\bar{w}=B^T\bar{z}
\end{cases}
$$
P is completly controllable $\leftrightarrow$ D is completly observable.
$$K_p^T=L_D$$

\section*{Canonical Forms}

\subsection*{Computed Form}
Assume that system $(A, b, c^T)$ is completely controllable, then it can be transformed into an $(\tilde{A}, \tilde{b}, \tilde{c}^T)$-system, where \\
$\tilde{A}=
\begin{pmatrix}
	0 & 0 & 0 & \dots & 0 & a_0 \\
	1 & 0 & 0 & \dots & 0 & a_1 \\
	0 & 1 & 0 & \dots & 0 & a_2 \\
	\vdots & \vdots & \vdots & \ddots & \vdots & \vdots \\
	0 & 0 & 0 & \dots & 1 & a_{n-1} 
\end{pmatrix}$ and 
$\tilde{b}=\begin{pmatrix}
1\\
0\\
0\\
\vdots\\
0
\end{pmatrix}$

$$K=(b,Ab,A^2b,\dots,A^{n-1}b)$$
$$T = K^{-1}$$
$$\tilde{A}=TAT^{-1}$$
$$\tilde{b}=Tb$$
$$\tilde{c}^T=C^TT^{-1}$$

\subsection*{Assembled Form}
$\tilde{A}=
\begin{pmatrix}
	0 & 1 & 0 & \dots & 0 \\
	0 & 0 & 1 & \dots & 0 \\
	\vdots & \vdots & \vdots & \ddots & \vdots \\
	0 & 0 & 0 & \dots & 1 \\
	a_0 & a_1 & a_2 & \dots & a_{n-1} 
\end{pmatrix}$ and 
$\tilde{b}=\begin{pmatrix}
0\\
0\\
\vdots\\
0\\
1
\end{pmatrix}$

$$K=(b,Ab,A^2b,\dots,A^{n-1}b)$$

$$\begin{pmatrix}
	t_1^T\\
	t_2^T\\
	\vdots\\
	t_n^T
\end{pmatrix} = K^{-1}$$

$$T=\begin{pmatrix}
	t_n^T\\
	t_n^TA\\
	t_n^TA^2\\
	\vdots\\
	t_n^TA^{n-1}\\
\end{pmatrix}$$

$$\tilde{A}=TAT^{-1}$$
$$\tilde{b}=Tb$$
$$\tilde{c}^T=C^TT^{-1}$$

\section*{Estimation and Design}
\subsection*{Theorem 9.1}
To find the matrix $K$ follow the algorithm below
\begin{itemize}
	\item Find the cannonical form of the system. 
\end{itemize}

\end{document}
