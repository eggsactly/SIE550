\documentclass[12pt]{article}
\usepackage{lingmacros}
\usepackage{tree-dvips}
\usepackage{amsmath}
\usepackage{accents}
\newcommand{\ubar}[1]{\underaccent{\bar}{#1}}
\usepackage{hyperref}
\hypersetup{
    colorlinks=true,
    linkcolor=blue,
    filecolor=magenta,      
    urlcolor=cyan,
}
\begin{document}

\section*{Definitions}

\begin{itemize}

\item Positive Definite: For function $V(\bar{x})>0, \forall x | x\neq0$ (Stability Concepts (I), Page 9).

\item Semi-Definite Negative: For function $V(\bar{x})\leq0$ (Stability Concepts (II), Page 3).

\item Hurowitz: When all eigenvalues of A are such that $\mathcal{R}_e(\lambda_i)<0$ A is called Hurowitz. 

\item Decrescent: $v(\bar{x},t)$ is decrescent if $v(\bar{0},t)=0$ and if $\exists$ \\ a time-invariant positive definite function $v_e(\bar{x})$ such that $\forall t\geq0$, $v(\bar{x},t)\leq v_e(\bar{x})$. 

\item Invariant: A set $G$ is an {\em invariant set} for a dynamic system if every system trajectory which starts in $G$ stays in $G$ for all future times. (Stability concepts (II) pages 1)

\item Limit Cycle: A limit cycle in the phase-plane of a second-order system is an isolated closed trajectory. 

\item Non-Singular: A matrix is non-singular if an inverse exists for the matrix. A square matrix is non-singular {\em if-and-only-if} its determinant is non-zero.

\item Inner Sum (Direct Sum): Represented with the symbol $\oplus$ is a coordinate-wise sum. For example consider the sets $a=(x_1,y_1)$ and $b=(x_2, y_2)$, $a\oplus b=(x_1+x_2,y_1+y_2)$.

\end{itemize}


\newpage
\section*{Stability Theorems}
\subsection*{Stability of Autonomous Systems, $\dot{\bar{x}}=\bar{f}(\bar{x})$}
\begin{itemize}
\item Stable (LaSalle theorem) (Stability Concepts (I), page 11): a time invariant system is {\em stable} when it meets the following criteria. There exists a {\em Lyapunov function} where:
	\begin{itemize}
	\item $V(\bar{x})>0, \forall x \in D/\{\bar{0}\}$
	\item $V(\bar{0})=\bar{0}$
	\item $V(\bar{x})\leq0, \forall x \in D$
	\end{itemize}
\item Asymptotically Stable if:
	\begin{itemize}
	\item The criteria for stable are met and
	\item $V(\bar{x})<0, \forall x \in D/\{\bar{0}\}$
	\end{itemize}
\item Globally Asymptotically Stable if:
	\begin{itemize}
	\item The criteria for Asymptotically Stable are met and
	\item $V(\bar{x})\to\infty$ as $||\bar{x}||\to\infty$
	\end{itemize}
\end{itemize}

\newpage
\subsection*{Stability of Non-Autonomous Systems, $\dot{\bar{x}}=\bar{f}(\bar{x},t)$}
\begin{itemize}
\item Stable (Barbalat's Lemma) (Stability Concepts (II), page 10): a time variant system is {\em stable} when it meets the following criteria. There exists a {\em Lyapunov function} where:
	\begin{itemize}
	\item $V(\bar{x},t)>0$ is positive definite
	\item $\dot{V}(\bar{x},t)$ is negative semi-definite 
	\end{itemize}
\item Uniformly Stable where:
	\begin{itemize}
	\item The Criteria for Stable are met and
	\item $V(\bar{x},t)$ is {\em Decrescent}
	\end{itemize}
\item Uniformly Asymptotically Stable
	\begin{itemize}
	\item The Criteria for Uniform Stability are met and
	\item $\dot{V}(\bar{x},t)<0,\forall x \in D/\{\bar{0}\}$
	\end{itemize}
\item Globally Uniformly Asymptotically Stable:
	\begin{itemize}
	\item The Criteria for Uniform Asymptotically Stable are met and
	\item $V(\bar{x})\to\infty$ as $||\bar{x}||\to\infty$
	\end{itemize}
\item Exponential Stability (Stability Concepts (II), page 20):
	\begin{itemize}
	\item $V(\bar{x},t)\geq\alpha(||\bar{x}||)>0$
	\item $\dot{V}(\bar{x},t)\leq-\gamma(||\bar{x}||)$
	\item $V(\bar{x},t)\leq\beta(||\bar{x}||)$
	\item such that:
		\begin{itemize}
		\item $\alpha(||\bar{x}||)=k_1||\bar{x}||^a$
		\item $\beta(||\bar{x}||)=k_2||\bar{x}||^a$
		\item $\gamma(||\bar{x}||)=k_3||\bar{x}||^a$
		\end{itemize}
	\end{itemize}
\item Globally Exponentially Stable
	\begin{itemize}
	\item Exponential Stability criteria apply and:
	\item $c_1||\bar{x}||^2\leq V(\bar{x},t)\leq c_2||\bar{x}||^2$
	\end{itemize}
\end{itemize}

\section*{Jordan Form}
\subsection*{Finding Eigenvectors For Defective Matricies}
Given that $\lambda$ is repeated $n$ times. $v_i$ can be found with:
$$det(A-\lambda I)\bar{v_1}=\bar{0}$$
$$det(A-\lambda I)\bar{v_2}=\bar{v_1}$$
$$det(A-\lambda I)\bar{v_3}=\bar{v_2}$$
And so on...
\subsection*{Finding Solution With Jordan Blocks}
If you have a Jordan Block, a solution will be in the form:
$$e^{J_it}=e^{(\lambda_iI+N_i)t}=e^{\lambda_it}e^{N_it}$$
Where $e^{\lambda_it}$ is a scalar value and $e^{N_it}$ is a matrix because $N_i$ is a matrix. $e^{N_it}$ can be found with the following equation.

$$e^{N_it}=\begin{bmatrix} 1 & t & \frac{t^2}{2} & \dots & \frac{t^{n-1}}{(n-1)!} \\
0 & 1 & t & \dots & \frac{t^{n-2}}{(n-2)!} \\
\vdots & \vdots & \vdots & \ddots & \vdots \\
0 & 0 & 0 & 1 & t \\
0 & 0 & 0 & 0 & 1\\
\end{bmatrix}$$ 

\end{document}
